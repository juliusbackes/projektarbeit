\documentclass[tikz]{standalone}

\tikzset{%
  rows/.style={
    row ##1/.append style={#2}
  },
  columns/.style={
    column ##1/.append style={#2}
  },
  row 1/.style={fill=red!20} % Example of styling the first row
}

\makeatletter
\tikzset{
  left delimiter/.style 2 args={append after command={% overwriting original
    \tikz@delimiter{south east}{south west}% second argument is optional
    {every delimiter,every left delimiter,#2}{south}{north}{#1}{.}{\pgf@y}}}}
\makeatother

\begin{document}
\begin{tikzpicture}[
  >=Latex, node distance=2cm,
  my matrix/.style={
    draw, matrix of nodes,
    nodes={
      node family/text width/.expanded=%
        \tikzmatrixname-\the\pgfmatrixcurrentcolumn,
      node family/text width align=left,
      inner xsep=+.5\tabcolsep, inner ysep=+0pt, align=left},
    inner sep=.5\pgflinewidth,
    font=\strut\ttfamily
  }
]

\matrix[
    my matrix, 
    nodes={draw, minimum height=1cm, text width=3cm}, 
    row 1/.style={nodes={fill=red!20}}
] (m1) {
  id    & userId     & name     & description   & examStartDate & examEndDate & graph \\
  int8  & uuid       & string   & string        & timestampz    & timestampz  & jsonb \\
};

\matrix[
    my matrix, 
    below=of m1, 
    nodes={draw, minimum height=1cm, text width=3cm}, 
    row 1/.style={nodes={fill=red!20}},
    column 9/.style={nodes={text width=5cm}}
] (m2) {
  id & name & projectId & is2xHJ & grade & studentCount & studentList & adjancencyList & possibleExamDates \\
  int8 & string & int & boolean & int & int & string[] & string[] & int[] \\
};

\node[draw, align=center, text width=5cm, font=\ttfamily, above=4cm] (authid) {users.id};

\node[above=1cm of m1, draw, align=center] (attrib) {Attribute};

\foreach \col in {1,2,3,4,5,6,7} {
    \draw[->] (attrib.south) -- (m1-1-\col.north);
}

\path[node distance=.25em, font=\bfseries] 
  node[west above=of m1] {Projects}
  node[west above=of m2] {Exams};

\draw (m2-1-3.north) -- node[midway, right, xshift=0.2cm] {Primary Key} (m1-2-1.south);

\draw (authid.south) -- (m1-1-2.north);

\end{tikzpicture}
\end{document}
