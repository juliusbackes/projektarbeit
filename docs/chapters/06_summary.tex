\newpage
\section{Zusammenfassung und Ausblick}
\subsection{Zusammenfassung}
In dieser Arbeit haben wir uns mit dem Problem der zeitlichen Optimierung des Klausurenplans befasst und dafür eine Web-Applikation entwickelt, die helfen soll, dieses Problem zu lösen. Ziel der Arbeit war es, eine Webanwendung zu erstellen, die mithilfe von Graphentheorie arbeitet und in kurzer Zeit unkompliziert konfliktfreie Klausurenpläne erstellt.\\\\
Im theoretischen Teil der Arbeit wurden Grundlagen zum Thema Graphentheorie, insbesondere zur Graphenfärbung, dargestellt. Es wurde erklärt, wie mithilfe dieser Methoden zeitliche Überschneidungen zwischen verschiedenen Klausuren vermieden werden können, sodass keine zwei Klausuren, die im Konflikt stehen, gleichzeitig stattfinden. Das Problem wurde in einem Konfliktgraphen abgebildet und durch Färben des Graphen die kleinste Anzahl an Zeiträumen ermittelt. Es wurden Grundlagen zur Programmierung vermittelt und erläutert, warum die Arbeit zum Bau einer Website von Bedeutung ist. Die eingesetzten Werkzeuge im Front- und Backend-Bereich wurden detailliert beschrieben. Der Kernalgorithmus dieser Arbeit ist Dsatur (Degree of Saturation), da er vergleichsweise einfach zu implementieren ist und liefert gute Ergebnisse bei kurzer Laufzeit, was ihn besonders für Anwendungen mit vielen Klausuren und Schülern geeignet macht.\\\\
Im praktischen Teil wurde die Web-Applikation entwickelt, die mithilfe von Graphentheorie in kurzer Zeit ausgeglichene Klausurenpläne erstellt. Die Anwendung verarbeitet vom Nutzer hochgeladene Excel-Dateien mit Schüler- und Kursdaten und erstellt daraus einen Klausurenplan. Die Applikation berücksichtigt dabei die Begrenzung der Klausuren pro Schüler und mit einer Klausur am Tag. Die Klausurtermine sind über den vom Nutzer gewählten Zeitraum gleichmäßig verteilt, um die Schüler zu entlasten. Bei der Gestaltung der Webanwendung stand die Benutzerfreundlichkeit im Vordergrund. Es wurde übersichtlich und minimalistisch gestaltet, um dem Nutzer ein angenehmes Arbeitserlebnis zu bieten.\\\\
Insgesamt bietet die Arbeit eine gute Grundlage für eine schnelle und konfliktfreie Klausurenplanung. Die entwickelte Web-App vereinfacht den Planungsprozess für Lehrkräfte und ermöglicht es, Klausurenpläne rechtzeitig zu erstellen. Dadurch wird der organisatorische Aufwand reduziert und die Vorbereitung für Schüler und Lehrkräfte erleichtert.
\subsection{Ausblick}
Obwohl die Implementierung unserer Webanwendung zur Klausurenplanung erfolgreich abgeschlossen wurde, gibt es noch einige Bereiche, die in Zukunft erweitert oder verbessert werden können. Sowohl die Funktionalität der Anwendung als auch die Benutzerfreundlichkeit bieten Potential für weitere Optimierungen.\\\\
Der verwendete Algorithmus liefert bereits gute Ergebnisse. Dennoch kann versucht werden, die Performance und die Qualität der Ergebnisse durch weitere sinnvolle Metriken zu verbessern. Ein Ansatz hierfür wäre ein noch schnellerer Algorithmus, der noch bessere Lösungen berechnet. Ebenso könnte versucht werden, sogenannte Cluster in den Graphen zu erkennen. Diese würden dann Profil- bzw. freie LKs und GKs repräsentieren. Darüber könnte man versuchen eine lokale Einfärbung durchzuführen, um noch bessere Einfärbungen zu erhalten.\\\\
Die Webanwendung kann derzeit nur für das Bundesland Bremen genutzt werden, da die eingetragenen Feiertage nicht für alle Bundesländer gelten. Um die Feiertage automatisch zu berücksichtigen, könnte die Anwendung um eine Funktion zur Auswahl der Region erweitert werden. Auf diese Weise könnte der Nutzer Klausurpläne für Schulen in jedem beliebigen Bundesland erstellen.\\\\
Eine zukünftige Version der Web-Applikation sollte die Möglichkeit bieten, dem Kalender eigene unterrichtsfreie Tage hinzuzufügen. So könnte der Nutzer Tage auswählen, an denen keine Prüfungen geschrieben werden dürfen, z.B. bei schulinternen Veranstaltungen wie Schüler-Eltern-Lehrer-Gesprächen oder einem pädagogischen Tag.\\\\
Bisher gibt es keine Möglichkeit, sich bei Fragen zur Bedienung oder bei Fehlern an uns zu wenden. Ein User-Support würde dies ermöglichen. Man füllt ein Kontaktformular aus und gibt an, was das Problem ist oder welche Fragen man hat. Wir könnten dann darauf eingehen und den Nutzern helfen, problemlos mit unserer Webanwendung zu arbeiten.\\\\
Diese Erweiterungen würden die Anwendung noch leistungsfähiger, flexibler und benutzerfreundlicher machen und dazu beitragen, den Prozess der Prüfungsplanung weiter zu optimieren.