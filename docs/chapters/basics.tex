\section{Grundlagen der Graphentheorie}
In diesem Kapitel führen wir grundlegende Konzepte der Graphentheorie ein. Diese sind essenziell, um ein fundiertes Verständnis der Funktionsweise der finalen Anwendung zu entwickeln.

\subsection{Grundbegriffe}
Ein Graph in der Graphentheorie weist signifikante Unterschiede zu Graphen in der Analysis auf. Ein graphentheoretischer Graph besteht aus einer Sammlung von Knoten (eng. vertices), die je nach Art des Graphs durch Kanten (eng. edges) miteinander verbunden sein können.
\begin{figure}[h]
    \centering
    \documentclass{standalone}

\usepackage{tikz, pgfplots}
\pgfplotsset{compat=1.8}

\begin{document}

\usetikzlibrary {graphs}
\tikz [nodes={draw, circle, fill=blue!70}, text=white]{
  \node (a) at (0,0) {$A$};
  \node (b) at (1,1) {$B$};
  \node (c) at (1,-1) {$C$};
  \node (d) at (2,0) {$D$};

  \graph { (a) -- {(b), (c)} -!- (d) };
}

\end{document}
    \caption{Ein einfacher ungerichteter Graph}
\end{figure}
\subsubsection{Was ist ein Graph?}
Ein Graph $G$ wird als ein Paar $G=(V_G,E_G)$ definiert, wobei \(V_G\) eine Menge von Knoten darstellt, z.B. \(\{1, 8, 7\}\) oder \(\{A, B, C\}\), und \(E_G\) eine Menge von Kanten ist, also Verbindungen zwischen je zwei Knoten. Für die Kantenmenge gilt:
$$E_G = \{\{x, y\} \colon \; x,y \in V \wedge x \neq y \}$$
Zur einfacheren Darstellung von Graphen verwenden wir folgende Notation:
$$\{\{a, b\}, \{c, d\}\} \equiv \{ab, cd\}$$
Demnach wird der Graph in Abbildung 1 mit \(G=(V,E)\) beschrieben, wobei \(V=\{A,B,C,D\}\) und \(E=\{AB,AC\}\) sind. 
\newpage
Bis jetzt haben wir uns nur den ungerichteten Graphen angesehen. Bei einem gerichteten Graphen werden statt Mengen geordnete Paare für die Darstellung von Kanten genutzt. 
$$E_G = \{(x, y) \colon \; x,y \in V \wedge x \neq y \}$$
\begin{figure}[h]
    \centering
    \documentclass{standalone}

\usepackage{tikz, pgfplots}
\pgfplotsset{compat=1.8}

\begin{document}

\usetikzlibrary {graphs}
\tikz [nodes={draw, circle, fill=green!40}]{
  \node (a) at (0,0) {$A$};
  \node (b) at (2,0.5) {$B$};
  \node (c) at (1,-1) {$C$};
  \node (d) at (2,-1) {$D$};

  \graph { (c) -> {(a), (b), (d)}, (d) -> (b) };
}

\end{document}
    \caption{Ein einfacher gerichteter Graph}
\end{figure}
Für das Beispiel in Abb. 2 würde sich dann $G=(V_G, E_G)$ mit $V_G=\{A,B,C,D\}$ und $E_G=\{(C,A), (C,B), (C,D), (D,B)\}$ ergeben. Im Folgenden werden nur noch ungerichtete Graphen von Nöten sein, da gerichtete Graphen nicht nötig für die Anwendung dieser Arbeit sind. 
\begin{definition}
    Sei $G=(V_G, E_G)$ ein Graph und $a, b \in V_G$, so nennt man $a$ und $b$ angrenzend gdw. $\{a,b\} \in E_G$ gilt.
\end{definition}
\begin{definition}
    Der \textbf{Grad} eines Graphen $G$, ist die Anzahl seiner Knoten. 
    $$\deg(G)=|V_G|$$
\end{definition}
\begin{definition}
    Die \textbf{Größe} eines Graphen $G$, ist die Anzahl seiner Kanten; sie ergibt sich aus $|E_G|$.
\end{definition}
\begin{definition}
    Als einen \textbf{Empty-Graph} bezeichnet man einen Graphen $G$ mit $V_G \neq \{\} \wedge E_G=\{\}$.\\
\end{definition}
\begin{definition}
     Als einen \textbf{Null-Graph} bezeichnet man einen Graphen $G$ mit $V_G=\{\} \wedge E_G=\{\}$.
\end{definition}
Einen Graphen $X$ mit $V_X = \{\} \; \wedge \; E_X =\{\}$ kann es nicht geben, da $\{a,b\} \in V_X$ gelten muss.
\newpage
\subsubsection{Teilgraphen}
Ein Teilgraph ist ein Graph $T$ eines Graphen $G$ bei dem folgendes gilt
$$V_T \subseteq V_G \wedge E_T \subseteq E_G \Rightarrow T \subseteq G$$