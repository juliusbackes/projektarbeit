\newpage
\section{Einleitung}
Die Erstellung eines ausbalancierten Klausurenplans stellt eine große Herausforderung dar, da verschiedene Anforderungen berücksichtigt werden müssen. In diesem Kapitel gehen wir auf die Problemstellung, die Motivation für die Arbeit und die Ziele des Projektes ein.

\subsection{Problemstellung}
Die Planung der Klausuren in der Oberstufe ist aufgrund der Vielzahl an Fächern und Kursen äußerst komplex und zeitaufwendig. Dies führt häufig zu einer unausgewogenen Verteilung der Klausuren. Der Klausurenplan muss dabei bestimmten Regeln folgen: So darf ein Schüler nicht mehr als eine Klausur pro Tag und maximal drei Klausuren pro Woche schreiben. Idealerweise sollte ein Schüler sogar weniger als drei Klausuren pro Woche schreiben.

\subsection{Motivation}
In den letzten Jahren haben wir selbst während unserer Zeit in der Oberstufe der Oberschule an der Ronzelenstraße erfahren, wie schwierig es ist, einen ausgewogenen Klausurenplan zu erstellen. Die Erstellung ist für die zuständigen Lehrkräfte mit einem erheblichem Zeitaufwand verbunden. Häufig wurde der Klausurenplan erst eine Woche vor Beginn der Klausurenphase veröffentlicht. Für uns Schüler war dies sehr frustrierend, da dies bedeutete, dass man erst spät mit Vorbereitung auf die Klausuren beginnen konnte.

\subsection{Ziel der Arbeit}
Um in Zukunft die o. g. Probleme zu vermeiden und den Prozess der Klausurplanung zu optimieren, entwickeln wir im Rahmen dieser Projektarbeit eine Web-App, die folgende Anforderungen erfüllt: Aus einer Excel-Datei, die die Kurse und Schüler beinhaltet und vom Nutzer hochgeladen wird, soll ein Klausurenplan für die eingegebenen Kurse erstellt werden. Zudem sollen die folgenden Anforderungen an den Klausurenplan sichergestellt werden:

\begin{enumerate}
    \item Kurse, in denen zwei Klausuren pro Halbjahr vorgesehen sind, sollen entsprechend umgesetzt werden.
    \item Ein Schüler darf maximal eine Klausur pro Tag schreiben und keine Klausuren zur gleichen Zeit haben.
    \item Ein Schüler darf maximal drei Klausuren pro Woche schreiben.
\end{enumerate}
