\newpage
\section{Einleitung}
Die Erstellung eines ausbalancierten Klausurenplans stellt eine große Herausforderung dar, da verschiedene Anforderungen berücksichtigt werden müssen. In diesem Kapitel gehen wir auf die Problemstellung, die Motivation für die Arbeit und die allgemeinen Ziele des Projektes ein.

\subsection{Problemstellung}
Die Planung von Klausuren in der Oberstufe ist aufgrund verschiedener Faktoren äußerst komplex. Die große Anzahl von Fächern und Kursen erschwert die praktische Planung.
Dies führt häufig zu einer unausgewogenen Verteilung der Klausuren. Ein fertiger Klausurenplan muss dabei bestimmten Regeln folgen. Unter anderem darf ein Schüler nicht mehr als eine Klausur pro Tag und maximal drei Klausuren pro Woche schreiben. Im Idealfall sollte ein Schüler sogar weniger als drei Klausuren pro Woche schreiben.

\subsection{Motivation}
Wir selbst haben in den letzten Jahren während unserer Zeit in der Oberstufe der Oberschule an der Ronzelenstraße erfahren, wie schwierig es ist, einen ausgewogenen Klausurenplan zu erstellen. Die Erstellung ist für die zuständigen Lehrkräfte mit einem erheblichen Zeitaufwand verbunden. So kommt es häufiger vor, dass der Klausurenplan erst eine Woche vor Beginn der Klausurenphase veröffentlicht wird. Dies kann sowohl für die Lehrkräfte als auch für die Schüler problematisch sein, da beide Parteien genügend Zeit benötigen, um sich auf eine Klausur vorzubereiten; die Lehrkräfte müssen bis zum Klausurtermin genügend Inhalte präsentiert haben und die Schüler müssen genügend Zeit zum Lernen haben. Dies ist nur mit ausreichendem Vorlauf möglich. Daher ist es wichtig, dass die Termine rechtzeitig bekannt sind.

\subsection{Ziel der Arbeit}
Um die o. g . Probleme in Zukunft zu vermeiden und den Prozess der Klausurplanung zu optimieren, entwickeln wir im Rahmen dieser Projektarbeit eine Webapplikation. Ziel ist die einfache Erstellung eines Klausurenplans, an den folgende Anforderungen gestellt werden.

\begin{enumerate} 

\item Kurse, in denen zwei Klausuren pro Halbjahr vorgesehen sind, werden entsprechend umgesetzt. 

\item Ein Schüler schreibt maximal eine Klausur pro Tag. 

\item Ein Schüler schreibt maximal drei Klausuren pro Woche. 

\end{enumerate}
Zur einfachen Bedienung wird eine vom Benutzer hochgeladene Excel-Datei, die die Kurse und Schüler enthält, verarbeitet, um automatisch einen Klausurenplan zu erstellen.
