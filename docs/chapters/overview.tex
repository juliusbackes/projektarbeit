\thispagestyle{plain}
\newpage
\noindent\textbf{Julius Backes \& Tom Kurzke}\\\\
\textbf{Thema}\\
Zeitliche Optimierung der
Klausurenplanung an Schulen
mithilfe von Graphentheorie — eine
Web-App-Entwicklung\\\\\\
\textbf{Leitfrage}\\
Wie lässt sich mithilfe von Graphentheorie und der Entwicklung einer Web-App die Klausurenplanung an Schulen zeitlich optimieren?\\\\\\
\textbf{Stichworte}\\
Graphentheorie, Graph Coloring, DSatur, Klausurenplanung\\\\
\textbf{Kurzzusammenfassung}\\
Diese Arbeit behandelt die automatisierte Planung von Prüfungsterminen für Schülerinnen und Schüler mithilfe eines graphentheoretischen Ansatzes. Durch den angepassten Dsatur-Algorithmus wird versucht, eine möglichst optimale Verteilung der Klausuren über mehrere Wochen zu erreichen, wobei schulische und organisatorische Anforderungen berücksichtigt werden.
