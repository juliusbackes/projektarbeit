\newpage
\section{Entwicklung der Web-Applikation}
Die Entwicklung der Webanwendung verfolgt das Ziel, einen effizienten und automatisierten Prozess für die Klausurenplanung in der Oberstufe zu schaffen. In diesem Kapitel werden die Anforderungen definiert und die technische Umsetzung beschrieben.
\subsection{Anforderungen und Spezifikationen}
Für eine strukturierte Umsetzung müssen die funktionalen und nicht-funktionalen Anforderungen an die Anwendung definiert werden. Schlussendlich soll im Rahmen der technischen Möglichkeiten eine übersichtliche und benutzerfreundliche Oberfläche geschaffen werden. Eine frühzeitige Differenzierung der Anforderungen erleichtert im allgemeinen die spätere Umsetzung und strukturierte Evaluation.
\subsubsection{Funktionale Anforderungen}
Funktionale Anforderungen definieren, welche konkreten Funktionen und Aufgaben eine Anwendung erfüllen soll. Sie beschreiben, welche Leistungen das System erbringen muss, um die gewünschten Ziele zu erreichen \parencite{anforderungen}. Im Folgenden werden wir die funktionalen Anforderungen in Kernfunktionen und Benutzerfunktionen unterteilen.\\\\
\textbf{Kernfunktionen}
\begin{itemize}
    \item Die Web-Applikation soll automatisiert einen Klausurenplan im Excel-Format erstellen, basierend auf den vom Benutzer übermittelten Daten.

    \item Der Klausurenplan sollte die Anzahl der verschiedenen Klausurtage minimieren und dabei die folgenden Regeln berücksichtigen:
    \begin{itemize}
        \item Maximal drei Klausuren pro Schüler pro Woche.
        \item Maximal eine Klausur pro Schüler und Tag.
        \item Keine Klausuren an Feiertagen oder während der Schulferien.
    \end{itemize}

    \item Die Web-Anwendung sollte den Klausurenplan und die dazugehörigen Graphen visualisieren können.
\end{itemize}
\textbf{Benutzerfunktionen}
\begin{itemize}
    \item Dem User soll die Möglichkeit gegeben werden, folgende Daten hochzuladen und auszuwählen:
    
    \begin{itemize}
        \item Der User soll eine Kursliste im Excel-Format herunterladen können. Diese soll alle Kurse mit ihren Schüler enthalten. Daraus soll für jeden Kurs eine Liste erstellt werden, welche Kurse nicht gleichzeitig mit dem Kurs stattfinden können.
        
        \item Der User sollte Kurse auswählen können, die zweimal pro Halbjahr eine Klausur schreiben.
        
        \item Der User sollte, jedem Kurs mindestens einen oder mehrere mögliche Prüfungstage (Wochentage) zuzuweisen können.
        
        \item Der User sollte einen Zeitraum festlegen können, in dem alle Klausuren stattfinden sollen. Diesen Zeitraum nennen wir Klausurenphase.
    \end{itemize}
    
    \item Alle o. g. Daten sollten von dem Benutzer geändert werden können. 
    
    \item Es sollte möglich sein, die Klausurenpläne in verschiedenen Projekten zu strukturieren, um eine bessere Übersicht zu gewährleisten. 
    \item Die Projekte sollten nur für den Ersteller zugänglich sein.
\end{itemize}
\subsubsection{Nicht-funktionale Anforderungen}
Nichtfunktionale Anforderungen umfassen alle Anforderungen, die nicht die Funktion, sondern die Qualität des Endprodukts bestimmen \parencite{anforderungen}.
\begin{itemize}
    \item Die Web-App soll performant sein und auch bei vielen Kursen und Schülern schnell einen Klausurenplan erstellen.
    \item Das Design der Webanwendung sollte modern, minimalistisch und konsistent sein.
    \item Auf die Benutzerfreundlichkeit ist zu achten. Funktionen sollten nicht versteckt, sondern leicht verständlich und intuitiv  sein.
    \item Der Quellcode der Webanwendung sollte sinnvoll strukturiert, fehlerfrei und wiederverwendbar sein.
    \item Moderne Web-Technologien sollen eingesetzt werden, um eine robuste und zukunftssichere Entwicklung zu gewährleisten.
\end{itemize}
\subsection{Systemarchitektur}
Die Systemarchitektur umfasst alle wichtigen Technologien und Werkzeuge, die zur Implementierung der Webanwendung verwendet werden. Sie beschreibt die grundlegende Struktur der Anwendung, einschließlich der Trennung zwischen Frontend und Backend und deren Interaktion. Das Frontend ist für die Benutzeroberfläche zuständig, während das Backend die Datenverarbeitung und Logik übernimmt \parencite{mccartney-2024}. Die Kommunikation zwischen beiden Bereichen erfolgt über standardisierte Schnittstellen. So genannte Application Programming Interfaces (APIs) \parencite{gazarov-2019}. Ziel der Architektur ist es, eine skalierbare, wartbare und performante Anwendung zu schaffen, die effizient weiterentwickelt werden kann.
\subsubsection{Frontend-Design}
Das Frontend-Design befasst sich mit der Benutzeroberfläche. Welche Werkzeuge werden verwendet, um dem Benutzer welche Funktionen auf welche Weise anzuzeigen? \parencite{mccartney-2024}\\\\
Das Grundgerüst einer Webseite wird mithilfe der Hypertext Markup Language (HTML) erstellt. Elemente können durch Cascading Style Sheets (CSS) mit Attributen wie Farbe, Textgröße und Ähnlichem versehen werden. Die Logik für Animationen sowie die Interaktion mit der Webseite wird über die Programmiersprache JavaScript (JS) gesteuert \parencite{mdn-getting-started-web}. Obwohl der Name es vermuten lassen könnte, besteht kein Zusammenhang zwischen JavaScript und der Programmiersprache Java \parencite{geeksforgeeks-2024}.\\\\
JavaScript wird in der Regel direkt im Browser ausgeführt. Es handelt sich um eine dynamisch typisierte Sprache, d.h. Variablen und Funktionen können ähnlich wie in Python ohne explizite Typenangaben definiert werden \parencite{mdn-javascript}. Dies kann allerdings zu einfachen Fehlern führen. Um dem entgegenzuwirken, wird in modernen Webanwendungen häufig TypeScript verwendet. TypeScript ist eine Erweiterung von JavaScript, bei der Typen explizit angegeben werden müssen. Nach der Entwicklung wird TypeScript in reguläres JavaScript kompiliert \parencite{typescript-tutorial-2024}.
\begin{minted}{javascript}
let x = 42;
x = "Hello, World!";
\end{minted}
In JavaScript würde der obige Code keinen Fehler verursachen, da die Variable \texttt{x} keinen festen Typ hat. In TypeScript hingegen würde bereits während der Entwicklung ein Fehler angezeigt werden, da die Variable \texttt{x} ursprünglich als \texttt{number} (Java-Äquivalent: \texttt{int} – ganzzahlige Zahl) deklariert wurde und daher nicht mit einem \texttt{string} (Zeichenfolge) überschrieben werden darf \parencite{typescript-tutorial-2024}.\\\\
In der modernen Webentwicklung nutzt man sogenannte Frameworks. Das sind Software-Bibliotheken, die Entwicklern Werkzeuge, Strukturen und vorgefertigte Funktionen bereitstellen, um die Entwicklung von dynamischen Webanwendungen effizienter und einfacher zu gestalten. Sie bieten eine standardisierte Basis für häufig benötigte Aufgaben wie Routing, die Bereitstellung von Webservern und die Nutzung von Reaktivität \parencite{mdn-intro-to-cs-frameworks}.\\\\
Bekannte Beispiele für Web-Frameworks sind React, Django, Angular und Svelte \parencite{mdn-intro-to-cs-frameworks}. Für dieses Projekt wird das open-source-web-framework Svelte verwendet, da es als modernes und robustes Framework bekannt ist. Im Vergleich zu vielen anderen Frameworks, die mit der Zeit zunehmend komplexer wurden, bietet Svelte einen einfachen und direkten Ansatz. Diese Komplexität bei anderen Frameworks entsteht häufig dadurch, dass neue Probleme durch wachsende Anforderungen und zusätzliche Funktionen gelöst werden mussten \parencite{svelte-rethinking}.\\\\
Svelte verfolgt jedoch einen anderen Ansatz, indem es die meisten Aufgaben bereits während der Kompilierung löst. Dadurch wird weniger JavaScript im Browser ausgeführt, was die Performance verbessert und die Entwicklungsarbeit erleichtert. Die einfache Syntax und die starke Fokussierung auf Benutzerfreundlichkeit machen Svelte besonders attraktiv für moderne Webprojekte \parencite{svelte-rethinking}.\\\\
tailwind - shadcn
\newpage
\subsubsection{Backend-Design}
Das Backend umfasst alle Prozesse, die im Hintergrund der Anwendung ablaufen. Es steuert die Logik für Benutzerfunktionen, verarbeitet Datenbankabfragen, regelt die Kommunikation zwischen Diensten und sorgt dafür, dass Funktionen wie Benutzerauthentifizierung reibungslos funktionieren \parencite{nam-le-thanh-web-designer-2023}.\\\\
Das Backend der hier entwickelten Webanwendung soll vor allem zwei Hauptaufgaben erfüllen. Erstens die Authentifizierung: Dazu gehören die Speicherung von Benutzerdaten, die Verschlüsselung von Passwörtern und die Speicherung von sogenannten Authentifizierungstokens, die für die Kommunikation mit dem Backend benötigt werden. Zweitens die Speicherung aller anderen Daten. Alle Kurse, Graphen und alle damit verbundenen Daten müssen gespeichert werden.\\\\
Dazu benötigt man eine Datenbank. In diesem Projekt wird eine SQL-Datenbank (Structured Query Language) verwendet, da es durch die feste Datenstruktur und die einfache Verwaltung von Beziehungen ideale Voraussetzungen für die sichere und einheitliche Speicherung komplexer Daten wie Kurse und Graphen bietet QUELLE.\\\\
Nun muss noch eine Lösung für die andere Hauptaufgabe gefunden werden. Die Antwort heißt Firebase bzw. Supabase. Firebase ist ein Backend-as-a-Service (Baas), das von Google entwickelt wurde. Es ist Cloud-basiert und hat viele wichtige Funktionen wie Datenbank, Authentifizierung und vieles mehr bereits integriert. Supabase ist eine Open-Source-Kopie von Firebase. Unternehmen wie Mozilla, PwC, Johnson \& Johnson und 1Password verwenden Supabase. https://supabase.com/ga.\\\\

sveltekit?\\\\\\\\
Alle Voraussetzungen sind erfüllt: Die Benutzeroberfläche wird mit TypeScript und dem Framework Svelte entwickelt. Im Backend übernimmt Supabase sowohl die Datenbankverwaltung als auch die Benutzerauthentifizierung und bildet damit eine stabile und skalierbare Grundlage für die Anwendung.
% \subsubsection{Technologie-Stack}
\subsection{Implementierung}
\subsubsection{Algorithmische Umsetzung}
\subsubsection{Benutzeroberfläche}